\section{Tree Structures}
%MARK: Binary Search Tree
\subsection{Binary Search Tree}
A binary search tree is a binary tree where each node has a key and the keys are ordered such that for each node, all keys in the left subtree are less than the key of the node and all keys in the right subtree are greater than the key of the node. 

\begin{algorithm}
    \floatname{algorithm}{Biary Search Tree}
    \caption{\textsc{Tree-Search(x, k)}: Go left if the key is less than the current node, go right if the key is greater than the current node. The average case time complexity is \( \mathcal{O}(\log n) \) and the worst case time complexity is \( \mathcal{O}(n) \).}
    \begin{algorithmic}
        \If{ x = NULL or  k = \text{key}[x]  }
            \State\Return{ x }
        \EndIf{}
        \If{  k < key[x]  }
            \State\Return{ \Call{Tree-Search}{left[x], k} }
        \Else{}
            \State\Return{ \Call{Tree-Search}{right[x], k} }
        \EndIf{}
    \end{algorithmic}
\end{algorithm}
    
\begin{algorithm}
    \floatname{algorithm}{Biary Search Tree}
    \caption{\textsc{TreeMin(x)}: Go left until you reach the leftmost node. The time complexity is \( \mathcal{O}(\log n) \) with a worst case time complexity of \( \mathcal{O}(n) \).}
    \begin{algorithmic}
        \If{ left[x] = NULL or x = NULL }
            \State\Return{ x }
        \EndIf{}
        \State\Return{ \Call{TreeMin}{left[x]} }
    \end{algorithmic}
\end{algorithm}

\begin{algorithm}
    \floatname{algorithm}{Biary Search Tree}
    \caption{\textsc{TreeSucc(x)}: If the right subtree is not empty, return the minimum of the right subtree. Otherwise, go up the tree until you reach a node that is the left child of its parent. The time complexity is \( \mathcal{O}(\log n) \) with a worst case time complexity of \( \mathcal{O}(n) \).}
    \begin{algorithmic}
        \If{ right[x] \( \neq \) NULL }
            \State\Return{ \Call{TreeMin}{right[x]} }
        \EndIf{}
        \State{ \( y \gets p[x] \)}
        \While{ y \( \neq \) NULL and x = right[y] }
            \State{ \( x \gets y \) }
            \State{ \( y \gets p[y] \) }
        \EndWhile{}
        \State\Return{ y }
    \end{algorithmic}
\end{algorithm}

\begin{algorithm}
    \floatname{algorithm}{Biary Search Tree}
    \caption{\textsc{TreeInsert(x, z)}: Insert a node into the binary search tree. The time complexity is \( \mathcal{O}(\log n) \) with a worst case time complexity of \( \mathcal{O}(n) \). The idea is to traverse the tree until we reach a leaf node and then insert the new node there.}
    \begin{algorithmic}
        \State{ \( y \gets NULL \) }
        \State{ \( x \gets root[T] \) }
        \While{ x \( \neq \) NULL }
            \State{ \( y \gets x \) }
            \If{ key[z] < key[x] }
                \State{ \( x \gets \) left[x]  }
            \Else{}
                \State{ \( x \gets \) right[x] }
            \EndIf{}
        \EndWhile{}
        \State{ p[z] = y }
        \If{ y = NULL }
            \State{ root[T] = z }
        \ElsIf{ key[z] < key[y] }
            \State{ left[y] = z }
        \Else{}
            \State{ right[y] = z }
        \EndIf{}
    \end{algorithmic}
\end{algorithm}

\begin{algorithm}
    \floatname{algorithm}{Biary Search Tree}
    \caption{\textsc{Delete(x, z)}: Delete a node from the binary search tree. The time complexity is \( \mathcal{O}(\log n) \) with a worst case time complexity of \( \mathcal{O}(n) \). If the node has no children or only one child, we can simply remove the node. If the node has two children, we can replace it with its successor.}
    \begin{algorithmic}
        \If{ left[z] = NULL or right[z] = NULL }
            \State{ \( y \gets z \) }
        \Else{}
            \State{ \( y \gets \) \Call{TreeSucc}{z} }
        \EndIf{}
        \If{ left[y] \( \neq \) NULL }
            \State{ \( x \gets \) left[y] }
        \Else{}
            \State{ \( x \gets \) right[y] }
        \EndIf{}
        \If{ x \( \neq \) NULL }
            \State{ p[x] = p[y] }
        \EndIf{}
        \If{ p[y] = NULL }
            \State{ root[T] = x }
        \ElsIf{ y = left[p[y]] }
            \State{ left[p[y]] = x }
        \Else{}
            \State{ right[p[y]] = x }
        \EndIf{}
        \If{ y \( \neq \) z }
            \State{ key[z] = key[y] }
        \EndIf{}
        \State\Return{ y }
    \end{algorithmic}
\end{algorithm}

%MARK: Red-Black Trees
\subsection{Red-Black Trees}
A red-black tree is a binary search tree with the following properties:
\begin{itemize}
    \item The height is at most \( \mathcal{O}(\log n) \)
    \item The black-height of a node is the number of black nodes on the path from the node to the leaves, not counting the node itself. The black-height of the tree is the black-height of the root.
    \item A red-black tree of black-height \( h \) has at least \( 2^h - 1 \) internal nodes. 
    \item The root is black
    \item Every leaf is black
    \item For each node \( x \), all simple paths from \( x \) to descendant leaves have the same number of black nodes
    \item Every red node has two black children
\end{itemize}

\begin{itemize}
    \item Properties:
        \begin{itemize}
            \item Every node is either red or black
            \item The root is always black
            \item Every leaf (NIL) is black
            \item If a node is red, then both its children are black (no consecutive red nodes)
            \item For each node, all simple paths from the node to descendant leaves contain the same number of black nodes (black height)
        \end{itemize}
    \item Insert operation:
        \begin{itemize}
            \item Insert the node as in a regular BST
            \item Color the new node red
            \item Fix violations of Red-Black properties:
                \begin{itemize}
                    \item Case 1: Uncle is red - Recolor parent, uncle, and grandparent
                    \item Case 2: Uncle is black (triangle) - Rotate parent
                    \item Case 3: Uncle is black (line) - Rotate grandparent and recolor
                \end{itemize}
            \item Ensure root remains black
        \end{itemize}
    \item Delete operation:
        \begin{itemize}
            \item Perform standard BST deletion
            \item If deleted node is black, fix double-black violation:
                \begin{itemize}
                    \item Case 1: Sibling is red - Recolor and rotate
                    \item Case 2: Sibling is black with black children - Recolor sibling
                    \item Case 3: Sibling is black with near child red - Rotate sibling's child and recolor
                    \item Case 4: Sibling is black with far child red - Rotate sibling and recolor
                \end{itemize}
        \end{itemize}
    \item Search operation:
        \begin{itemize}
            \item Standard BST search, no rebalancing needed
            \item Time complexity is \( \mathcal{O} \left( \log n \right) \) due to guaranteed height balance
        \end{itemize}
    \end{itemize}

%MARK: Splay Trees
\subsection{Splay Trees}

A splay tree is a self-adjusting binary search tree with the property that recently accessed elements are quick to access again. It performs basic operations such as insertion, look-up, and deletion in \( \mathcal{O}(\log n) \) amortized time. The idea is to move the accessed node to the root of the tree by performing a sequence of rotations.

%splay algorithm
\begin{algorithm}
    \floatname{algorithm}{Splay Tree}
    \caption{\textsc{Splay(x)}: Splay the node \( x \) to the root of the tree. The time complexity is \( \mathcal{O}(\log n) \) amortized.}
    \begin{algorithmic}
        % Case 1: x is the root
        % Case 2 (zig): x is the child of the root
        % Case 3 (zig-zig): x and x's parent are both left or right children
        % Case 4 (zig-zag): x is a left child and x's parent is a right child or vice versa
        \While{ x \( \neq \) root }
            \If{ parent(x) = root }
                \State{ \Call{Zig}{x} right rotation (or left rotation) }
            \ElsIf{ x is left child and parent(x) is left child or x is right child and parent(x) is right child }
                \State{ \Call{ZigZig}{x} right rotation grandparent, right rotation parent (or left-left rotation) }
            \Else{}
                \State{ \Call{ZigZag}{x} left rotation parent, right rotation grandparent (or right-left rotation) }
            \EndIf{}
        \EndWhile{}
    \end{algorithmic}
\end{algorithm}

\begin{itemize}
    \item For each operation (insert, search, delete), splay the node to the root of the tree
    \item If the node is not found, splay the last accessed node
    \item Delete operation:
        \begin{itemize}
            \item Search for x
            \item Splay(x)
            \item Search for predecessor y of x. By definition this is the largest element in the left subtree of x
            \item splay (y) on left subtree of x. By definition this will result in a left subtree with no right child
            \item Join the right subtree of x as the child of y
        \end{itemize}
    \item Insert operation:
        \begin{itemize}
            \item Search for x
            \item y is the last accessed node, which is either the predecessor or successor of x
            \item splay(y)
            \item insert x as the root of the tree. If y is the predecessor of x it becomes the left child, otherwise it becomes the right child.
        \end{itemize}
    \item Search operation:
        \begin{itemize}
            \item Search for x
            \item splay(x), or the last accessed node if x is not found
        \end{itemize}
\end{itemize}

%MARK: Skip lists


%MARK: van Emde Boas Trees

%MARK: Hashing

%MARK: Cuckoo Hashing