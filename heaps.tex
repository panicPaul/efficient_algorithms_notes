\section{Heaps}

%MARK: prio-q
\subsection{Priority Queue}

\paragraph{Dijkstra's Algorithm} (Single-Source Shortest Path)
The algorithm finds the shortest path from a source vertex to all other vertices in a weighted graph. For a vertex set \( V \) and edge set \( E \):


\begin{align*}
& \text{Initialize:} \\
& \quad d[s] = 0 \text{ (source distance)} \\
& \quad d[v] = \infty \text{ (all other vertices)} \\
& \quad Q = V \text{ (priority queue)} \\
\\
& \text{While } Q \neq \emptyset: \\
& \quad u = \text{extract\_min}(Q) \\
& \quad \text{For each neighbor } v \text{ of } u: \\
& \quad \quad \text{If } d[v] > d[u] + w(u,v): \\
& \quad \quad \quad d[v] = d[u] + w(u,v)
\end{align*}
    


Time complexity: \( O((|V| + |E|)\log|V|) \) with binary heap

\paragraph{Prim's Algorithm} (Minimum Spanning Tree)
Builds a minimum spanning tree by selecting edges with minimum weight that connect a vertex in the tree to a vertex outside the tree. For a graph \( G=(V,E) \):


\begin{align*}
& \text{Initialize:} \\
& \quad \text{key}[s] = 0 \text{ (source vertex)} \\
& \quad \text{key}[v] = \infty \text{ (all other vertices)} \\
& \quad \text{parent}[v] = \text{NIL} \text{ (for all vertices)} \\
\\
& \text{While priority queue } Q \neq \emptyset: \\
& \quad u = \text{extract\_min}(Q) \\
& \quad \text{For each } v \in \text{Adj}[u]: \\
& \quad \quad \text{If } v \in Q \text{ and } w(u,v) < \text{key}[v]: \\
& \quad \quad \quad \text{parent}[v] = u \\
& \quad \quad \quad \text{key}[v] = w(u,v)
\end{align*}


Time complexity: \( O(|E|\log|V|) \) with binary heap

Key differences:
1. Dijkstra's maintains distances from source: \( d[v] \)
2. Prim's maintains weights of edges: \( \text{key}[v] \)
3. Dijkstra's finds shortest paths from a source
4. Prim's builds a minimum spanning tree

The MST weight is given by:
\begin{equation*}
    MST_{weight} = \sum_{v \in V \setminus \{s\}} w(v, \text{parent}[v])
\end{equation*}

% MARK: Binary Heaps 
\subsection{Binary-Heap}
Complete binary tree, except for the last level with the property that the parent node is always smaller than their children.
\begin{itemize}
    \item By definition the minimum element is the root
    \item The maximum element is one of the leaves
    \item Insertion of \( x \):
        \begin{itemize}
            \item Insert \( x \) as leftmost leave element.
            \item Exchange \( x \) with parent untill the min-property is restored
        \end{itemize}
    \item Deletion of \( x \):
        \begin{itemize}
            \item Replace \( x \) with the rightmost leave element \( y \)
            \item Fix min property of \( y \)
        \end{itemize}
\end{itemize}

Let \( A[1, \cdots,  n] \) be an array representing a binary heap. Then:
\begin{itemize}
    \item The parent of element \( i \) can be found in \( A \left[ \left\lfloor \frac{i}{2}  \right\rfloor \right] \)
    \item The left child of element \( i \) can be found at in \( A \left[ 2i + 1 \right] \)
    \item The right child of element \( i \) can be found at in \( A \left[ 2i + 2 \right] \)
\end{itemize}

%MARK: Binomial Heap
\subsection{Binomial Heap}



%MARK: Fibonacci Heap
\subsection{Fibonacci Heap}
\begin{itemize}
    \item collection of min-heap trees
    \item root level has a doubly linked circular list
    \item root level has a min pointer to the minimum element. Since each tree is a min-heap, the min pointer is the minimum of all trees
    \item We keep track of wether the nodes are marked or not
\end{itemize}

\paragraph{Insert}
\begin{itemize}
    \item Insert a new node with key \( k \) into the root list
    \item If the new node is smaller than the current min, update the min pointer
\end{itemize}

\paragraph{Delete-Min}
\begin{itemize}
    \item Remove the minimum element (pointed to by min pointer)
    \item Add all its children to the root list
    \item Consolidate the root list:
        \begin{itemize}
            \item Combine trees of the same degree until no two trees have the same degree
            \item Create a temporary array indexed by tree degree
            \item For each tree in root list, combine with existing tree of same degree if any
            \item When combining, make the larger root a child of the smaller root
        \end{itemize}
    \item Update the min pointer by finding minimum among remaining roots
\end{itemize}

\paragraph{Delete}
\begin{itemize}
    \item Decrease the key to negative infinity
    \item Cut the node from its parent and add it to the root list
    \item If parent is not root, mark it
    \item If parent was already marked:
        \begin{itemize}
            \item Cut parent from its parent and add to root list
            \item Continue cascading cuts up the tree until reaching either:
                \begin{itemize}
                    \item An unmarked node (mark it and stop)
                    \item A root (stop)
                \end{itemize}
        \end{itemize}
    \item Perform Delete-Min operation to remove the node with negative infinity key
\end{itemize}

\paragraph{Decrease-Key}
\begin{itemize}
    \item Decrease the key of the node to the new value
    \item If new key is smaller than parent's key:
        \begin{itemize}
            \item Cut the node from its parent and add to root list
            \item Perform cascading cuts on the parent as in Delete operation
        \end{itemize}
    \item If new key is smaller than current minimum, update min pointer
\end{itemize}